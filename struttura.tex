%%%%%%%%%%%%%%%%%%%%%%%%%%%%%%%%%%%%%%%%%%%%%%%%%%
%
% Per la generazione corretta del 
% pdflatex nome_file.tex
% bibtex nome_file.aux
% pdflatex nome_file.tex
% pdflatex nome_file.tex
%
%%%%%%%%%%%%%%%%%%%%%%%%%%%%%%%%%%%%%%%%%%%%%%%

% formato FRONTE RETRO
%\usepackage{epsfig}

\usepackage{plain}
\usepackage{setspace}

%--------------------------------------------------------------------
%------------------------- C H I A R A  layout ----------------------
% Better page layout for A4 paper, see memoir manual.
\settrimmedsize{297mm}{210mm}{*}
\setlength{\trimtop}{0pt} 
\setlength{\trimedge}{\stockwidth} 
\addtolength{\trimedge}{-\paperwidth} 
\settypeblocksize{634pt}{448.13pt}{*} 
\setulmargins{4cm}{*}{*} 
\setlrmargins{*}{*}{*} 
\setmarginnotes{17pt}{51pt}{\onelineskip} 
\setheadfoot{\onelineskip}{2\onelineskip} 
\setheaderspaces{*}{2\onelineskip}{*} 
\checkandfixthelayout
\setlength{\oddsidemargin}{0.35cm}%
\setlength{\evensidemargin}{-0.35cm}%
%
\frenchspacing
% Font with math support: New Century Schoolbook
\usepackage{fouriernc}
\usepackage[T1]{fontenc}
% Text should be in double or 1.5 line spacing, and font size should be
% chosen to ensure clarity and legibility for the main text and for any
% quotations and footnotes. Margins should allow for eventual hard binding.
%
% Note: This is automatically set by memoir class. Nevertheless \OnehalfSpacing 
% enables double spacing but leaves single spaced for captions for instance. 
\OnehalfSpacing 
%
% Sets numbering division level
\setsecnumdepth{subsection} 
\maxsecnumdepth{subsubsection}
%
\usepackage{float}						%Helps to place figures, tables, etc. 
%\newfloat{algo}{tbp}{loa}[subparagraph*]

\usepackage{calc,soul,fourier}
\makeatletter 
\newlength\dlf@normtxtw 
\setlength\dlf@normtxtw{\textwidth} 
\newsavebox{\feline@chapter} 
\newcommand\feline@chapter@marker[1][4cm]{%
	\sbox\feline@chapter{% 
		\resizebox{!}{#1}{\fboxsep=1pt%
			\colorbox{white}{\color{black}\thechapter}% 
		}}%
		\rotatebox{0}{% 
			\resizebox{%
				\heightof{\usebox{\feline@chapter}}+\depthof{\usebox{\feline@chapter}}}% 
			{!}{\scshape\so\@chapapp}}\quad%
		\raisebox{\depthof{\usebox{\feline@chapter}}}{\usebox{\feline@chapter}}%
} 
\newcommand\feline@chm[1][4cm]{%
	\sbox\feline@chapter{\feline@chapter@marker[#1]}% 
	\makebox[0pt][c]{% aka \rlap
		\makebox[0cm][r]{\usebox\feline@chapter}%
	}}
	
\makechapterstyle{daleifmodif}{
	\renewcommand\chapnamefont{\normalfont\Large\scshape} 
	\renewcommand\chaptitlefont{\normalfont\Large\bfseries\scshape} 
	\renewcommand\chapternamenum{} \renewcommand\printchaptername{} 
	\renewcommand\printchapternum{\null\hfill\feline@chm[1.5cm]} 
	\renewcommand\afterchapternum{\par\vskip\midchapskip} 
	\renewcommand\printchaptertitle[1]{\color{black}\chaptitlefont ##1}
} 
\makeatother 
\chapterstyle{daleifmodif}
%\usepackage{fancyhdr}
%\pagestyle{fancy} \mainmatter \linespread{1.1} \pagestyle{fancy}
%\renewcommand{\chaptermark}[1]{\markboth{#1}{}} \fancyhf{}
%\renewcommand{\footrulewidth}{0pt}
%\renewcommand{\headrulewidth}{0pt}
%
%\fancypagestyle{plain}{%
%\fancyhead{} % leva l'intestazione
%\renewcommand{\headrulewidth}{0pt} % e la linea
%
%\mainmatter \linespread{1.0} \pagestyle{fancy}
%\renewcommand{\chaptermark}[1]{\markboth{#1}{}}
%\fancyhead[LE,RO]{\bfseries\thepage}
%\fancyhead[RE]{\hspace{40pt}\small\mdseries\leftmark}
%\fancyhead[LO]{\small\mdseries\leftmark\hspace{40pt}}
%
%
% The pages should be numbered consecutively at the bottom centre of the
% page.
\makepagestyle{myvf} 
\makeoddfoot{myvf}{}{\thepage}{} 
\makeevenfoot{myvf}{}{\thepage}{} 
\makeheadrule{myvf}{\textwidth}{\normalrulethickness} 
\makeevenhead{myvf}{\small\textsc{\leftmark}}{}{} 
\makeoddhead{myvf}{}{}{\small\textsc{\rightmark}}
\pagestyle{myvf}
%
% Oscar's command (it works):
% Fills blank pages until next odd-numbered page. Used to emulate single-sided
% frontmatter. This will work for title, abstract and declaration. Though the
% contents sections will each start on an odd-numbered page they will
% spill over onto the even-numbered pages if extending beyond one page
% (hopefully, this is ok).
\newcommand{\clearemptydoublepage}{\newpage{\thispagestyle{empty}\cleardoublepage}}
%
%
% Creates indexes for Table of Contents, List of Figures, List of Tables and Index
\makeindex

%%%%%%%%%%%%%%%%%%%%%%%%%%%%%%%%%%%%%%%%%%%%%%%%%%%%%%%%%%%%%%%%
% Numerazione romana- con caratteri piccoli nell'indice
%%%%%%%%%%%%%%%%%%%%%%%%%%%%%%%%%%%%%%%%%%%%%%%%%%%%%%%%%%%%%%%%
\makeatletter 
\renewcommand{\frontmatter}{% 
  \cleardoublepage\@mainmatterfalse 
  \pagenumbering{smallRoman}} 
\DeclareRobustCommand{\@smrc}[1]{% 
  \begingroup\check@mathfonts\fontsize{\sf@size}{\sf@size}% 
  \selectfont#1\endgroup} 
\def\smallRoman#1{\expandafter\@smallRoman\csname c@#1\endcsname} 
\def\@smallRoman#1{\@smrc{% 
  \expandafter\@slowromansmallcap\romannumeral #1@}} 
\def\@slowromansmallcap#1{\ifx @#1% then terminate 
  \else 
    \if i#1I\else 
    \if v#1V\else 
    \if x#1X\else 
    \if l#1L\else 
    \if c#1C\else 
    \if d#1D\else 
    \if m#1M\else 
  #1\fi\fi\fi\fi\fi\fi\fi 
  \expandafter\@slowromansmallcap 
\fi} 
%\pdfstringdefDisableCommands{\let\@smrc\@iden} 
%\makeatother

% \printglossaries below creates a list of abbreviations. \gls and related
% commands are then used throughout the text, so that latex can automatically
% keep track of which abbreviations have already been defined in the text.
%
% The import command enables each chapter tex file to use relative paths when
% accessing supplementary files. For example, to include
% chapters/brewing/images/figure1.png from chapters/brewing/brewing.tex we can
% use
% \includegraphics{images/figure1}
% instead of
% \includegraphics{chapters/brewing/images/figure1}
\usepackage{import}

% Add other packages needed for chapters here. For example:
\usepackage{lipsum}					%Needed to create dummy text
\usepackage{amsfonts} 					%Calls Amer. Math. Soc. (AMS) fonts
\usepackage[centertags]{amsmath}			%Writes maths centred down
\usepackage{stmaryrd}					%New AMS symbols
\usepackage{amssymb}					%Calls AMS symbols
\usepackage{amsthm}					%Calls AMS theorem environment
\usepackage{newlfont}					%Helpful package for fonts and symbols
\usepackage{layouts}					%Layout diagrams
\usepackage{graphicx}					%Calls figure environment
\usepackage{longtable,rotating}			%Long tab environments including rotation. 
%\usepackage[applemac]{inputenc}			%Needed to encode non-english characters 
									%directly for mac
%\usepackage{colortbl}					%Makes coloured tables
\usepackage{wasysym}					%More math symbols
\usepackage{mathrsfs}					%Even more math symbols
\usepackage{verbatim}					%Permits pre-formated text insertion
\usepackage{upgreek }					%Calls other kind of greek alphabet
\usepackage{latexsym}					%Extra symbols
\usepackage[square,numbers,
		     sort&compress]{natbib}		%Calls bibliography commands 
\usepackage{url}						%Supports url commands
\usepackage{etex}						%eTeXÕs extended support for counters
\usepackage{fixltx2e}					%Eliminates some in felicities of the 
									%original LaTeX kernel
\usepackage[english]{babel}		%For languages characters and hyphenation
\usepackage{color}                    				%Creates coloured text and background
%\usepackage[table]{xcolor}
\usepackage[colorlinks=true,
		     allcolors=black]{hyperref}              %Creates hyperlinks in cross references
\usepackage{memhfixc}					%Must be used on memoir document 
									%class after hyperref
\usepackage{enumerate}					%For enumeration counter
\usepackage{footnote}					%For footnotes
\usepackage{microtype}					%Makes pdf look better.
\usepackage{rotfloat}					%For rotating and float environments as tables, 
									%figures, etc. 
\usepackage{alltt}						%LaTeX commands are not disabled in 
									%verbatim-like environment
\usepackage[version=0.96]{pgf}			%PGF/TikZ is a tandem of languages for producing vector graphics from a 
\usepackage{tikz}						%geometric/algebraic description.
\usetikzlibrary{arrows,shapes,snakes,
		       automata,backgrounds,
		       petri,topaths}				%To use diverse features from tikz		
%							
%Reduce widows  (the last line of a paragraph at the start of a page) and orphans 
% (the first line of paragraph at the end of a page)
\widowpenalty=1000
\clubpenalty=1000
%


%--------------------------------------------------------------------

%\usepackage{titlesec} % per formato custom dei titoli dei capitoli

%%%%%%%%%%%%%%
% supporto lettere accentate
%\usepackage[utf8]{inputenc} % per Linux (richiede il pacchetto unicode);
\usepackage{filecontents}
\usepackage[english]{babel}
%
%\usepackage{natbib}
%\usepackage{bibentry}


%\singlespacing

%---------------------------------------------------------------------------------------------------
%	COLLEGAMENTI IPERTESTUALI

\usepackage{hyperref}
	\newcommand{\mail}[1]{\href{mailto:#1}{\texttt{#1}}}
%---------------------------------------------------------------------------------------------------
%	COLORI 
\definecolor{verde}{rgb}{0.09, 0.45, 0.27}
\definecolor{rosso}{rgb}{0.82, 0.1, 0.26}
\definecolor{blu}{rgb}{0.2, 0.2, 0.6}
%\usepackage[table]{xcolor}
%	\definecolor{mygreen}{RGB}{28,172,0}
%	\definecolor{mylilas}{RGB}{170,55,241}
%	\definecolor{mywithe}{RGB}{255,255,240}
%	\definecolor{wtblue}{RGB}{0,102,204}
%	\definecolor{lightgray}{gray}{0.9}
%---------------------------------------------------------------------------------------------------
%	GESTIONE IMMAGINI ED ELEMENTI ESTERNI 

\usepackage{wallpaper}
\usepackage{subfig}
\usepackage{graphicx}
\usepackage{tikz}
\usepackage{pgfplots}
	\pgfplotsset{/pgf/number format/use comma, compat=newest}
	\usetikzlibrary{plotmarks}
	\usetikzlibrary{datavisualization}
	\usetikzlibrary{intersections}
	\pgfkeys{
		/pgfplots/linelabel/.style args={#1:#2:#3}{
			name path global=labelpath,
			execute at end plot={
				\path [name path global=labelpositionline]
					(rel axis cs:#1,0) -- (rel axis cs:#1,1);
				\draw [
					help lines,
					text=black,
					inner sep=0pt,
					name intersections={
						of=labelpath and labelpositionline
					}]
					(intersection-1) -- +(#2) node [label={#3}] {};
			}
		}
	}
%----------------------------------------------------------
%Spostare foto
\usepackage{wrapfig}
\usepackage{sidecap}
%-----------------------------------------------------------------
% UNITA DI MISURA
\usepackage[output-decimal-marker={,}]{siunitx}
%----------------------------------------------------------------
%Interlinea elenchi
%INterlinea Elenchi
\usepackage{enumitem}
\setlist[enumerate]{nosep}
\setlist[itemize]{nosep}
\setlist[description]{nosep}
%--------------------------------------------------------------------------
%Tabelle
\usepackage{booktabs}
\usepackage{multirow}
\usepackage{caption}
\usepackage{longtable}
\captionsetup{tableposition=bottom,figureposition=bottom,font=small}
%-----------------------------------------------------------------
%Testo casuale
\usepackage{lipsum}
%----------------------------------------------------------------
%Formato titolo
\usepackage{titlesec} % per formato custom dei titoli dei capitoli
\titleformat{\chapter}
        {\normalfont\huge\bfseries}{\thechapter}{1em}{}

      \titlespacing*{\chapter}{0pt}{0in}{0.1in}
      \titlespacing*{\section}{0pt}{0.20in}{0.1in}
      \titlespacing*{\subsection}{0pt}{0.10in}{0.08in}
      \titlespacing*{\paragraph}{5mm}{0.02in}{5mm}
      \titlespacing*{\subsubsection}{0pt}{0.02in}{0in}

%\titleformat{\chapter}[hang]
%  {\normalfont\huge\bfseries}
%  {\chaptertitlename\ \thechapter:\ }
%  {0pt}
%  {\filcenter}

%\titlespacing*{\loa}{\normalfont\Large\bfseries\scshape}
\titlespacing*{\chapter}{0pt}{-20pt}{20pt}
\titleformat*{\chapter}{\normalfont\huge\bfseries\scshape}
\titleformat*{\section}{\Large\bfseries\scshape}
\titleformat*{\subsection}{\large\bfseries\scshape}

\titlespacing*{\subparagraph}{200pt}{50pt}{20pt}
\titleformat*{\subparagraph}{\large\bfseries\scshape\color{gray}}

\usepackage{etoolbox}
\makeatletter
\patchcmd{\chapter}{\if@openleft\cleardoublepage\else\clearpage\fi}{\par}{}{}
\makeatother

%BIBLIOGRAFIA
%\usepackage[autostyle, italian=guillemets]{csquotes}
%\usepackage{biblatex}
%\usepackage{guit} 
%\addbibresource{biblio.bib} 
%----------------------------------------------------------
\usepackage{indentfirst} % per rientrare la prima riga del primo capoverso dopo il titolo del paragrafo o del capitolo
\usepackage{amsmath} % per scrivere testo nelle formule
\usepackage[section]{placeins} % forza le immagini a comparire nella sezione in cui sono state generate e non nella successiva (cosa che succede se ad esempio manca spazio)
%\usepackage{afterpage}

% note a margine del testo
\newcommand{\sidenote}[1]{\marginpar[\raggedleft \small \itshape#1]{\small \itshape#1}} % note a margine piccole ed in corsivo

%------------------------------------------------------------------
%BIBLIO
\usepackage{url}
\usepackage{natbib}
\usepackage{bibentry}
\usepackage[resetlabels]{multibib}
\usepackage{cite}
\newcites{web}{Sitografia}
%--------------------------------------------------------------
%footnote a piè pagina
\usepackage[bottom]{footmisc}
%--------------------------------------------------------------
%freccia che gira con scritta interna
\usepackage{amssymb}% http://ctan.org/pkg/amssymb
%---------------------------------------------------------
%pacchetto per scrivere vari simboli come il permille
\usepackage[T1]{fontenc}
%-----------------------------------------
%definizione robusta per scrivere il permille sia che sia dentro la modalità testo sia che sia dentro la modalità matematica
\usepackage{textcomp}
%\usepackage{amsmath}
%\DeclareRobustCommand{\perthousand}{%
%  \ifmmode
%    \text{\textperthousand}%
%  \else
%    \textperthousand
%  \fi}
\usepackage{cancel}		% per semplificare le espressioni
%-------------------------------------------
%creare i frame
%\usepackage{showframe}
%-------------------------------------------
%per importare pdf per l'appendice
\usepackage{pdfpages}
%---------------------------------------------------------------------------------------------------
% COMANDI LIBRERIE DISEGNO
%
%
\usepackage{tikz}
\usetikzlibrary{datavisualization}
\usetikzlibrary{datavisualization.formats.functions}
\usepackage{psfrag}
\setcounter{tocdepth}{3}
\usetikzlibrary{decorations.pathreplacing}
\newcommand{\minitab}[2][l]{\begin{tabular}#1 #2\end{tabular}}
\usepackage{cancel}
\newcommand\Ccancel[2][black]{\renewcommand\CancelColor{\color{#1}}\cancel{#2}}
\usepackage{amsmath}
\usetikzlibrary{decorations.markings}
\usetikzlibrary{decorations.pathmorphing,patterns} %molle
\usetikzlibrary{plotmarks}
\usetikzlibrary{arrows,shapes,calc}


%---------------------------------------------------------------------------------------------------
%	COMANDI ABBREVIATI
%
\newcommand{\nb}{\textbf{N.B.:~}}
\newcommand{\pmi}{\tcperthousand}
\def\c{{\circ}}

\usepackage{enumitem}
\setitemize{label=\raisebox{.5\height}{\scalebox{0.6}{\textbullet}}}
\captionsetup{width=12.5cm}
\setlist[itemize]{itemsep=-0.5mm, topsep=1mm}

\newcommand{\ned}{N_{Ed}}
\newcommand{\med}{M_{Ed}}
\newcommand{\nrd}{N_{Rd}}
\newcommand{\mrd}{M_{Rd}}
\newcommand{\ved}{V_{Ed}}
\newcommand{\vrd}{V_{Rd}}

\newcommand{\figname}{Fig.}
\newcommand{\tabname}{Tab.}
\newcommand{\chapname}{Capitolo}
\newcommand{\secname}{Sezione}
\newcommand{\parname}{Paragrafo}
\newcommand{\eqname}{Eq.ne}
\newcommand{\appname}{Appendix}



\newcommand{\ntc}{\textit{NTC18}}
\newcommand{\ec}{\textit{Eurocodice~2}}
\newcommand{\cir}{\textit{Circ.~2019}}
\newcommand{\nb}{\textbf{N.B.}}


%---------------------------------------------------------------------------------------------------
%	COMANDI TABELLE
%
\usepackage{booktabs}
\usepackage{longtable}


%---------------------------------------------------------------------------------------------------
%	COMANDI MATLAB


\usepackage{listings}
%\lstset{escapeinside={<@}{@>}}
\lstset{morecomment=[s][\color{gray}]{/'}{'/}}
%\usepackage{xcolor}
%\lstdefinestyle{base}{
%language=Matlab,%
%basicstyle=\small\ttfamily,
%numbers=left,
%numberstyle=\tiny,
%stepnumber=2,
%frame=lines,
%moredelim=**[is][\color{gray}]{@}{@}
%}

% \RequirePackage{mcode}
% \usepackage[framed,numbered,autolinebreaks,useliterate]{mcode}
% \addto\captionsitalian{% 
% \renewcommand{\lstlistingname}{Codice}}

\usepackage{etoolbox}
%How to reduce space before and after listings
\newlength\savedparskip
\BeforeBeginEnvironment{lstlisting}{\setlength\savedparskip{\parskip}}
\lstset{belowskip=\dimexpr-\savedparskip+\medskipamount\relax}
\lstset{aboveskip=\dimexpr-\savedparskip+\medskipamount\relax}
%%How to avoid numbering in lines in listing
%%\makeatletter
%\lstset{numbers=left,escapeinside=||}
%\let\origthelstnumber\thelstnumber
%\newcommand*\Suppressnumber{%
%  \lst@AddToHook{OnNewLine}{%
%    \let\thelstnumber\relax%
%     \advance\c@lstnumber-\@ne\relax%
%    }%
%}
%\makeatother
%modifiche
%\makeatletter
%\patchcmd{\lst@MakeCaption}{{lol}{lstlisting}}{{lol}{\lstlistingtype}}{}{}
%\patchcmd{\lst@MakeCaption}{{lol}{lstlisting}}{{lol}{\lstlistingtype}}{}{}
%
%\newcommand\lstlistofClistings{%
%  \cleardoublepage\phantomsection
%  \pdfbookmark{Listati MATLAB}{MATLAB}
%  \begingroup
%  \let\l@Flstlisting\@gobbletwo
%  \let\l@Mlstlisting\l@lstlisting
%  \def\lstlistlistingname{Listati MATLAB}
%  \@ifundefined{@donefirst}{\@fileswfalse\global\let\@donefirst\@empty}{}
%  \lstlistoflistings
%  \endgroup
%}
%\makeatother
%\def\lstlistingtype{lstlisting}
%\lstnewenvironment{Mlstlisting}[1][]
%  {\def\lstlistingtype{Mlstlisting}\lstset{#1}}
%  {}
\definecolor{carmine}{rgb}{0.59, 0.0, 0.09}
\definecolor{dimgray}{rgb}{0.41, 0.41, 0.41}
\def\boxit#1{%
  \smash{\color{carmine}\fboxrule=1.15pt\relax\fboxsep=2pt\relax%
  \llap{\rlap{\fbox{\vphantom{0}\makebox[#1]{}}}~}}\ignorespaces
}
%---------------------------------------------------------------------------------------------------
%	COMANDI CHIARA
%
\newcommand{\ds}{\displaystyle}
\newcommand{\norme}{\textit{Norme Tecniche delle Costruzioni}}
\newcommand{\para}{\textit{Par.}}
\newcommand{\circo}{\textit{Circolare~Ministeriale}}
\usepackage{rotating}
\newcommand{\cm}{\checkmark}
\newcommand{\os}{\textit{OpenSees}}
\newcommand{\ml}{\textit{MATLAB}}
\newcommand{\bw}{\textit{Bouc-Wen}}
\newcommand{\dm}{\textit{DM}}
\newcommand{\edp}{\textit{EDP}}
\newcommand{\im}{\textit{IM}}
\newcommand{\dv}{\textit{DV}}
\newcommand{\teta}{\textbf{$\theta$}}